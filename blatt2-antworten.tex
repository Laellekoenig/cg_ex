\documentclass{article} %% Bestimmt die allgemeine Formatierung der Abgabe.
\usepackage{a4wide} %% Papierformat: A4.
\usepackage[utf8]{inputenc} %% Datei wird im UTF-8 Format geschrieben.
%% Unter Windows werden Dateien je nach Editor nicht in diesem Format
%% gespeichert und Umlaute werden dann nicht richtig erkannt.
%% Versucht in diesem Fall "utf8" auf "latin1" (ISO 8859-1) wechseln.
\usepackage[T1]{fontenc} %% Format der Zeichen im erstellten PDF.
%\usepackage[german]{babel} %% Regeln für automatische Worttrennung.
\usepackage{fancyhdr} %% Paket um einen Header auf jeder Seite zu erstellen.
\usepackage{lastpage} %% Wird für "Seite X von Y" im Header benötigt.
                      %% Damit das funktioniert, muss pdflatex zweimal
                      %% aufgerufen werden.
\usepackage{enumerate} %% Hiermit kann der Stil der Aufzählungen
                       %% verändert werden (siehe unten).

\usepackage{amssymb} %% Definitionen für mathematische Symbole.
\usepackage{amsmath} %% Definitionen für mathematische Symbole.
\usepackage{amsthm}

\usepackage{stmaryrd}

\usepackage{tikz}  %% Paket für Grafiken (Graphen, Automaten, etc.)
\usetikzlibrary{automata} %% Tikz-Bibliothek für Automaten
\usetikzlibrary{arrows}   %% Tikz-Bibliothek für Pfeilspitzen

%% Linke Seite des Headers
\lhead{\course\\\semester\\Übungsblatt \homeworkNumber}
%% Rechte Seite des Headers
\rhead{\authorname\\Seite \thepage\ von \pageref{LastPage}}
%% Höhe des Headers
\usepackage[headheight=36pt]{geometry}
%% Seitenstil, der den Header verwendet.
\pagestyle{fancy}

\newcommand{\authorname}{Max Jappert, Maximilian Barth}
\newcommand{\semester}{Frühjahrssemester 2021}
\newcommand{\course}{Computergrafik}
\newcommand{\homeworkNumber}{2}


\usepackage[T1]{fontenc}
%\usepackage{inconsolata}

\usepackage{color}

\definecolor{dkgreen}{rgb}{0,0.6,0}
\definecolor{gray}{rgb}{0.5,0.5,0.5}
\definecolor{mauve}{rgb}{0.58,0,0.82}

\lhead{\course\\\semester\\Übungsblatt \homeworkNumber}
\rhead{\authorname\\Seite \thepage\ von \pageref{LastPage}}

\usepackage{listings}
\lstset{frame=tb,
  language=Java,
  aboveskip=3mm,
  belowskip=3mm,
  showstringspaces=false,
  columns=flexible,
  basicstyle={\small\ttfamily},
  numbers=none,
  numberstyle=\tiny\color{gray},
  keywordstyle=\color{blue},
  commentstyle=\color{dkgreen},
  stringstyle=\color{mauve},
  breaklines=true,
  breakatwhitespace=true,
  tabsize=3
}

\begin{document}
\section{Seperierbare Filter}
\subsection{2D Faltung}

Wir können den 2D-Kernel Algorithmus folgend grob in Code schreiben:\\ \\
Bild der Grösse $M \times N$ und Filtermaske der Grösse $k \times k$

\begin{lstlisting}

for (int i = 0; i < M; i++) {																					
	for (int j = 0; j < N; j++) {																			
			// ueber jedes Pixel im Bild iterieren -> N * M Iterierungen
		for (int p = 0; p < k; p++) {																			
			for (int q = 0; q < k; q++) {	
											
				// durch jeden Eintrag des Kernels iterieren -> k *k Iterierungen
								
				applyFilter(i, j, p, q);
			}
		}
	}
}
\end{lstlisting}

\subsection{1D Faltung}

Den 1-D Kernel kann folgend grob beschrieben werden, mit einem Bild derselben Grösse und einer Filtermaske der Grösse $k$.

\begin{lstlisting}
// zuerst die erste Faltung ausfuehren
for (int i = 0; i < M; i++) {													
	for (int j = 0; j < N; j++) {												
	// durch jedes Pixel des Bildes iterieren  -> M * N Iterierungen
		for (int p = 0; p < k; p++) {											
			// 1D Filter anwenden -> k Iterierungen
			applyFilter(i, j, p);
		}
	}
}

// dann die zweite Faltung auf das entstandene Bild
for (int i = 0; i < M; i++) {												
	for (int j = 0; j < N; j++) {												
		// durch jedes Pixel des Bildes iterieren
		for (int p = 0; p < k; p++) {											
			// 1D Filter anwenden -> k Operationen
			applyFilter(i, j, p);
		}
	}
}
\end{lstlisting}
Empirisch lässt sich die Laufzeit zeigen, indem wir am Beispiel der Unit-Tests die Iterationen zählen. Da es sich bei beiden Tests um dasselbe Bild handelt, ist bei beiden der Eingabewert $n = M \cdot N = 480 \cdot 500 = 240000$.

Bei der 1D-Faltung wird einmal horizontal und einmal vertikal gefiltert, wobei beide Male über das gesamte Bild, also $n$-Mal iteriert wird, wobei pro Iteration $k = 5$ mal über den Kernel iteriert wird. Dies entspricht gesamthaft $2 \cdot n \cdot k = 2 \cdot 240000 \cdot 5 = 2400000$ Iterationen. Daraus lässt sich eine Laufzeit von $\Theta (2nk)$ ableiten, was $\mathcal{O} (n)$, also einer linearen Laufzeit entspricht, da bei $\mathcal{O}$ Konstanten wie $2k$ ignoriert werden.

Bei der 2D-Faltung wird nur einmal über das gesamte Bild iteriert, dabei wird pro Iteration einmal über den Kernel iteriert, also $k \cdot k = 25$ Mal. Somit wird gesamthaft $nk^2 = 480000 \cdot 25 = 6000000$ mal iteriert. Daraus lässt sich die Laufzeit $\Theta (nk^2)$ ableiten, was auch einer linearen Laufzeit von $\mathcal{O} (n)$ entspricht. Da die Konstante $k^2$ für Werte $k > 2$ grösser ist als die Konstante $2k$ handelt es sich bei der 1D-Faltung um eine konstant optimierte Laufzeit.

\section{Filtern im Fourierraum}

Eine Faltung im Ortsraum  entspricht einer Multiplikation im Fourierraum, wie auch andersrum. Dies bedeutet, dass wir die Anwendung der Kernel anstatt durch eine Faltung im Ortsraum (wie in den Übungen implementiert) auch durch eine Multiplikation im Fourierraum implementieren könnten. Dafür müssten wir zuerst das Bild und den Kernel in den Fourierraum transformieren, dort die beiden Funktionen miteinander multiplizieren und die daraus resultierende Funktion zurück in den Ortsraum transportieren. Die absolute Laufzeit des 2DFFT beträgt $MN log(MN)$, somit beträge die absolute Laufzeit dieses Prozesses $2MN log(MN) + MN$. Die $\mathcal{O}$-Laufzeit bleibt gleich, da Konstanten ignoriert werden.

Es muss also gelten, dass die Laufzeit der Multiplikation im Fourierraum geringer ist, falls $2MN log(MN) + MN > MNk^2 \Leftrightarrow 3 > k^2$ gilt. Wenn also $k^2 < 3$ gilt, dann lohnt sich der Umweg über den Fourierraum aus einer laufzeitorientierten Perspektive, obwohl die $\mathcal{O}$-Notation dies nicht widerspiegelt.

\end{document}