\documentclass{article} %% Bestimmt die allgemeine Formatierung der Abgabe.
\usepackage{a4wide} %% Papierformat: A4.
\usepackage[utf8]{inputenc} %% Datei wird im UTF-8 Format geschrieben.
%% Unter Windows werden Dateien je nach Editor nicht in diesem Format
%% gespeichert und Umlaute werden dann nicht richtig erkannt.
%% Versucht in diesem Fall "utf8" auf "latin1" (ISO 8859-1) wechseln.
\usepackage[T1]{fontenc} %% Format der Zeichen im erstellten PDF.
%\usepackage[german]{babel} %% Regeln für automatische Worttrennung.
\usepackage{fancyhdr} %% Paket um einen Header auf jeder Seite zu erstellen.
\usepackage{lastpage} %% Wird für "Seite X von Y" im Header benötigt.
                      %% Damit das funktioniert, muss pdflatex zweimal
                      %% aufgerufen werden.
\usepackage{enumerate} %% Hiermit kann der Stil der Aufzählungen
                       %% verändert werden (siehe unten).

\usepackage{amssymb} %% Definitionen für mathematische Symbole.
\usepackage{amsmath} %% Definitionen für mathematische Symbole.
\usepackage{amsthm}
\usepackage{gensymb} %% Für das Grad-Zeichen

\usepackage{stmaryrd}

\usepackage{tikz}  %% Paket für Grafiken (Graphen, Automaten, etc.)
\usetikzlibrary{automata} %% Tikz-Bibliothek für Automaten
\usetikzlibrary{arrows}   %% Tikz-Bibliothek für Pfeilspitzen

%% Linke Seite des Headers
\lhead{\course\\\semester\\Übungsblatt \homeworkNumber}
%% Rechte Seite des Headers
\rhead{\authorname\\Seite \thepage\ von \pageref{LastPage}}
%% Höhe des Headers
\usepackage[headheight=36pt]{geometry}
%% Seitenstil, der den Header verwendet.
\pagestyle{fancy}

\newcommand{\authorname}{Max Jappert, Maximilian Barth}
\newcommand{\semester}{Frühjahrssemester 2021}
\newcommand{\course}{Computergrafik}
\newcommand{\homeworkNumber}{3}


\usepackage[T1]{fontenc}
%\usepackage{inconsolata}

\usepackage{color}

\definecolor{dkgreen}{rgb}{0,0.6,0}
\definecolor{gray}{rgb}{0.5,0.5,0.5}
\definecolor{mauve}{rgb}{0.58,0,0.82}

\lhead{\course\\\semester\\Übungsblatt \homeworkNumber}
\rhead{\authorname\\Seite \thepage\ von \pageref{LastPage}}

\usepackage{listings}
\lstset{frame=tb,
  language=Java,
  aboveskip=3mm,
  belowskip=3mm,
  showstringspaces=false,
  columns=flexible,
  basicstyle={\small\ttfamily},
  numbers=none,
  numberstyle=\tiny\color{gray},
  keywordstyle=\color{blue},
  commentstyle=\color{dkgreen},
  stringstyle=\color{mauve},
  breaklines=true,
  breakatwhitespace=true,
  tabsize=3
}

\begin{document}
\section{Euler-Winkel}

Jede 3D-Rotation lässt sich durch die chronologische Ausführung dreier Basisrotationen um die jeweiligen Achsen darstellen. Somit bieten die Euler-Winkel eine intuitive Methode zur Rotation im dreidimensionalen Raum.

Das wesentliche Problem besteht im \textit{Gimbal Lock}. Dieser entsteht, wenn ein Körper so rotiert wird, dass zwei Rotationsachsen übereinanderliegen, etwa wenn der Körper vom Ursprung aus an einer Achse um $90 \degree$ rotiert wird. In diesem Zustand beschreibt die Rotation der beiden übereinanderliegenden Achsen dieselbe Bewegung, was zur Folge hat, dass ein Freiheitsgrad verlorengeht und somit nicht mehr jede Rotation durch eine Aneinanderreihung dreier Basisrotationen beschrieben werden kann.

\section{Quaternionen}

Wir wissen aus der Vorlesung, dass $q_1 \cdot q_2 = (s_1, \vec{v_1}) \cdot (s_2, \vec{v_2}) = (s_1 s_2 - \vec{v_1} \vec{v_2}, s_1 \vec{v_2} + s_2 \vec{v_1} + \vec{v_1} \times \vec{v_2})$. Davon abgeleitet wissen wir, dass der um den Winkel $\theta$ um die Achse $\vec{n}$ rotierte Punkt $p = (0, \vec{p})$ mit dem Quaternion $r = (cos(\frac{\theta}{2}), sin(\frac{\theta}{2})\vec{n})$ beschrieben werden kann. Die Formel dazu entnehmen wir der Folie 27 aus der Vorlesung zu den Transformationen: \\\\$p' = qpq^{-1} = cos(\theta)\vec{p} + (1 - cos(\theta))\vec{n}(\vec{p} \cdot \vec{n}) + sin(\theta) (\vec{n} \times \vec{p})$. \\\\
Wir haben nun $q = (cos(\frac{\pi}{6}), sin(\frac{\pi}{6})\begin{pmatrix}
sin(\frac{\pi}{6})\\
cos(\frac{\pi}{6})\\
0
\end{pmatrix})$ und $p = (0, \begin{pmatrix}
1\\
3\\
4
\end{pmatrix})$ in Quaternionenschreibweise. Somit sieht die Rechnung folgendermassen aus: \\\\
$p' = qpq^{-1} = cos(\frac{\pi}{6})\vec{p} + (1 - cos(\frac{\pi}{6})\vec{n}(\vec{p} \cdot \vec{n}) + sin(\frac{\pi}{6}) (\vec{n} \times \vec{p}) = cos(\frac{\pi}{6}) \begin{pmatrix}
1\\
3\\
4
\end{pmatrix} + (1 - cos(\frac{\pi}{6})) sin(\frac{\pi}{6}) \begin{pmatrix}
sin(\frac{\pi}{6})\\
cos(\frac{\pi}{6})\\
0
\end{pmatrix} \cdot \begin{pmatrix}
sin^2(\frac{\pi}{6})\\
3sin(\frac{\pi}{6})cos(\frac{\pi}{6})\\
0
\end{pmatrix} + sin(\frac{\pi}{6}) \begin{pmatrix}
4sin(\frac{\pi}{6})cos(\frac{\pi}{6})\\
-4sin^2(\frac{\pi}{6})\\
3 sin(\frac{\pi}{6}) - cos(\frac{\pi}{6})
\end{pmatrix} = \dotso = \begin{pmatrix}
1\\
3\\
4
\end{pmatrix}$. \\\\\\
Daraus folgt, dass der rotierte Punkt mit dem ursprünglichen identisch ist.
\end{document}